\documentclass[11pt]{article}
\usepackage[slovak]{babel}
\usepackage[utf8]{inputenc}
\usepackage[T1]{fontenc}
\usepackage{geometry}
\usepackage{amsmath}
\usepackage{amsfonts}
\usepackage{amssymb}
\usepackage{mathrsfs}
\usepackage{enumitem}
\usepackage{multicol}
\usepackage{watermark}
\usepackage{framed}
\usepackage{shuffle}
\usepackage{listings}
\usepackage{xcolor}
\lstset { %
    language=C++,
    backgroundcolor=\color{black!5}, % set backgroundcolor
    basicstyle=\footnotesize,% basic font setting
}

\geometry{
  a4paper,
  top=25mm,
  bottom=18mm,
  right=21mm,
  left=22mm
}

\newcommand{\eps}{\varepsilon}

\pagestyle{empty}
\setlength{\parindent}{0pt}
\setlength{\parskip}{8pt}
\setlist{topsep=0pt}

\begin{document}
\begin{center}
{\Large\textbf{Report zo zimného semestra}}
\end{center}
\thispageheading{\vspace{-42pt}\begin{flushright}\begin{tabular}{l} Matúš Duchyňa \\ \today \end{tabular}\end{flushright}}

Naprogramoval som naivný backtracking algoritmus na zistenie, či má daný graf $G$ acyklické \textit{vrcholové} regulárne farbenie s $k$ farbami (či platí $a(g) \leq k$) ako súčasť knižnice ba-graphs. Algoritmus postupne generuje všetky regulárne farbenia a skontroluje, či sú acyklické. Kontrola acykliskosti skontroluje každú dvojicu farieb, či neexistuje cyklus obsahujúci vrcholy iba s týmito farbami.

Ak farbenie existuje pre $k$-regulárny graf $G = (V, E)$, algoritmus ho zvládne nájsť do pár sekúnd, pokiaľ $|V| + k \approx 18$ (okrem zopár vínimok, kde sa čas zvýši na pár minút). V prípade, že farbenie neexistuje pre $k$-regulárny graf $G = (V, E)$, algoritmus to zistí do pár sekúnd, pokiaľ $|V| + k \approx 13$.

Otestoval som, že pre stromy platí $a(G) = 2$ a že pre grafy $G$ s maximálnym stupňom vrchola $\Delta (G)$ platí: \\
1. $\Delta(G) = 3 \Rightarrow a(G) \leq 4$, \\
2. $\Delta(G) = 4 \Rightarrow a(G) \leq 5$, \\
3. $\Delta(G) = 5 \Rightarrow a(G) \leq 7$, \\
4. $\Delta(G) = 6 \Rightarrow a(G) \leq 12$.

Na nájdenie acyklického \textit{hranového} regulárneho farbenia grafu $G = (V, E)$ stačí skonštruovať line graf $L(G)$ a nájsť acyklické \textit{vrcholové} regulárne farbenie grafu $L(G)$.

Otestoval som Fiamčíkovu hypotézu, tj. pre $k$-regulárne grafy $G$ platí $a(G) \leq k+2$, a ak $k=3$ a zároveň $G$ nie je $K_4$ ani $K_{3, 3}$, tak $a(G) \leq 4$.


V letnom semestry spravím algoritmus založený na SAT solveri.
\end{document}
